\section{Solution approach using a CNN}
\label{sec:ansatz}
The most successful approach to image recognition based an machine learing algorithms is the use of \textit{Convolutional Neural Networks} (CNN).
CNNs can use multidimensional input arrays which is their main advantage compared to \enquote{normal} fully-connected neural networks for working with image data, 
as the information of the position of each pixel compared to its neighbors is not lost by flattening the pixel-values in an one dimensional array.
This is done by repeatedly applying weight-matrices of size $N_\text{kernal}\times M_\text{kernal}$ called \textit{kernals} on the image, while moving the kernal from one edge of the image to the other 
whereby the steps size (referring to the number of pixel-rows/ -columns that the kernal is moved by in each step) is called \textit{stride}.
This allows CNNs to learn geometric shapes within the images it is trained on.
Because we want to classify the images into the eleven flower species, we also add a small fully-connected neural network after the CNN structure, so that the final output of our
neural network are eleven scores for each image which describe how likely the image shows a flower(s) of each of the eleven species.
A subsequent final assignment to one of the species can be done by choosing the species with the highest score.

% loss and accuracy 

\subsection{First network and manual hyperparameter tuning}
Due to our preprocessing the shape of the input data is $200\times 200\times 3$, where the last dimension represents the three RGB color channels. 
We start by trying different structures for the network to get a basic of idea of which type of network would work for our problem.
This leds to a network with three convolutional layers, each followed by a MaxPooling layer to stabilize the performance of the network against slight displacement or rotation of certain shapes 
in different images. % padding ?!
MaxPooling layers take the maximal value of a $N_\text{pool}\times N_\text{pool}$ subspace of each of the two-dimensional arrays of the three-dimensional input and return it as the single output 
for each of those subspaces.
After this convolutional part of the network we add two hidden fully-connected layers before the output layer, because a second hidden layer seems to improve the result.

As \textbf{activation functions} we use the \textit{ReLU} function for all layers but the output layer, for which we use the \textit{softmax} function.
This choice of activation functions makes it possible to interpret the output scores of the network as real probabilities of the image showing a flower of the given species.

Regarding the \textbf{number of kernals} in each convolutional layer, an increasing number of kernals seems to work best and we choose to double the number of kernals in each layer.
For the MaxPooling layers we choose $\symbf{N_\text{pool}} = 2$ as any larger number decreases the image size too fast.
The combination of an increasing number and kernals with pooling layers after each convolutional layer effectively leads to a compression of the first two dimensions (height and width of the image) 
of the data while more information gets transformed into the third dimension (number of kernals or RGB channels in the input layer).

For an image size of $200\times 200$ the kernal must also be quadratic ($N_\text{kernal} = M_text{kernal}$) and a \textbf{kernal size} of $N_\text{kernal} = 4$ seems to work well.
We tried using non-unit \textbf{stride}, but the best result is achieved by using the (default) unit stride which means moving the kernal one pixel-row/ -column at a time over the image.

An overview of the network structure we find based on these manual hyperparameter variations can be found in \autoref{fig:first_model}. 

% adam and learning rate and batch size


\subsection{Sequential grid searches for hyperparameter tuning}


\subsection{Regularization}

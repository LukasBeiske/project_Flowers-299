\section{Summary}
\label{sec:summary}
To summarize, we find, that a CNN can be trained and optimized to correctly classify flower images into one of the eleven species we chose with an accuracy of almost $\SI{80}{\percent}$.
Looking at which species get more often confused with one another, it can be said that most of the more common confusions are reasonable in a sense, that a human classification by hand
probably confuses these subsets of species with one another at times, as well.
Comparing the performance with a simpler approach, like using the kNN algorithm based only on the colors of an image, shows the use of CNNs for this problem to be appropriate 
and it yields a considerable performance increase.

\documentclass[
  bibliography=totoc,     % Literatur im Inhaltsverzeichnis
  captions=tableheading,  % Tabellenüberschriften
  titlepage=firstiscover, % Titelseite ist Deckblatt
  fontsize=12pt
]{scrartcl}

% 1,5 Zeilenabstand
\usepackage[onehalfspacing]{setspace}

% Paket float verbessern
\usepackage{scrhack}

% Warnung, falls nochmal kompiliert werden muss
\usepackage[aux]{rerunfilecheck}

% unverzichtbare Mathe-Befehle
\usepackage{amsmath}
\usepackage{amssymb}
\usepackage{mathtools}

% Fonteinstellungen
\usepackage{fontspec}

% Wenn man andere Schriftarten gesetzt hat,
% sollte man das Seiten-Layout neu berechnen lassen
\recalctypearea{}

% Spracheinstellungen
\usepackage[english]{babel}

\usepackage[
  math-style=ISO,    % ┐
  bold-style=ISO,    % │
  sans-style=italic, % │ ISO-Standard folgen
  nabla=upright,     % │
  partial=upright,   % ┘
  warnings-off={           % ┐
    mathtools-colon,       % │ unnötige Warnungen ausschalten
    mathtools-overbracket, % │
  },                       % ┘
]{unicode-math}

% traditionelle Fonts für Mathematik
\setmathfont{Latin Modern Math}
\setmathfont{XITS Math}[range={scr, bfscr}]
\setmathfont{XITS Math}[range={cal, bfcal}, StylisticSet=1]

% Zahlen und Einheiten
\usepackage[
  locale=UK,                   % englische Einstellungen
  separate-uncertainty=true,   % immer Unsicherheit mit \pm
  per-mode=symbol-or-fraction, % / in inline math, fraction in display math
]{siunitx}

% richtige Anführungszeichen
\usepackage[autostyle]{csquotes}

% schöne Brüche im Text
\usepackage{xfrac}

% Standardplatzierung für Floats einstellen
\usepackage{float}
\floatplacement{figure}{htbp}
\floatplacement{table}{htbp}

% Floats innerhalb einer Section halten
\usepackage[
  section, % Floats innerhalb der Section halten
  below,   % unterhalb der Section aber auf der selben Seite ist ok
]{placeins}

% Seite drehen für breite Tabellen: landscape Umgebung
\usepackage{pdflscape}

% Captions
\usepackage{caption}

% subfigure, subtable, subref
\usepackage{subcaption}

% Grafiken können eingebunden werden
\usepackage{graphicx}

% schöne Tabellen
\usepackage{booktabs}

% Verbesserungen am Schriftbild
\usepackage{microtype}

% Ein wenig mehr Farbe
\usepackage{xcolor}
\xdefinecolor{tugreen}{RGB}{132, 184, 25}

\addtokomafont{title}{\color{tugreen}}
\addtokomafont{section}{\color{tugreen}}
\addtokomafont{pagenumber}{\color{tugreen}}
%\addtokomafont{sectionentry}{\color{tugreen}}
\DeclareCaptionFont{tugreen}{\color{tugreen}}
\captionsetup{%
  labelfont={bf, tugreen},        % Tabelle x: Abbildung y: ist jetzt fett und in TU Grün
  font=small,                     % Schrift etwas kleiner als Dokument
  width=0.9\textwidth,            % maximale Breite einer Caption schmaler
}


% Literaturverzeichnis
\usepackage[
  backend=biber,
]{biblatex}
% Quellendatenbank
\addbibresource{lit.bib}
\addbibresource{programme.bib}

% Hyperlinks im Dokument
\usepackage[
  unicode,        % Unicode in PDF-Attributen erlauben
  pdfusetitle,    % Titel, Autoren und Datum als PDF-Attribute
  pdfcreator={},  % ┐ PDF-Attribute säubern
  pdfproducer={}, % ┘
]{hyperref}
% erweiterte Bookmarks im PDF
\usepackage{bookmark}

% Trennung von Wörtern mit Strichen
\usepackage[shortcuts]{extdash}

\author{%
  Lukas Beiske (report author)\\%
  \href{mailto:lukas.beiske@udo.edu}{lukas.beiske@udo.edu}%
  \and
  Lucas Cremer (project partner)\\%
  \href{mailto:lucas.cremer@udo.edu}{lucas.cremer@udo.edu}%
  \and%
  Daniel Wall (project partner)\\%
  \href{mailto:daniel.wall@udo.edu}{daniel.wall@udo.edu}%
}
\publishers{TU Dortmund – Fakultät Physik}
